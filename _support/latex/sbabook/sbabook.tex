\documentclass[english,twoside,openany,showtrims]{sbabook}

\pagelayout{\paperbackpage}

\usepackage[useregional]{datetime2}

% extract info from git commit
% requires gitinfo2 2.0.7, and below fix needed until 2.0.8 released
\makeatletter
\let\THEDAY\@dtm@currentday
\let\THEMONTH\@dtm@currentmonth
\let\THEYEAR\@dtm@currentyear
\makeatother
\usepackage[local,dirty=*]{gitinfo2}
\usepackage{xstring}% \IfEq and \StrCut
% a couple extensions...
\makeatletter
\newcommand\gitCommitInfo{%
  \IfEq{\gitRel}{}
  {commit \texttt{\gitAbbrevHash\gitDirty}}
  {\IfEq{\gitRoff}{0}
    {release \gitRel}
    {modified since release \gitRel{} --- commit \texttt{\gitAbbrevHash\gitDirty}}}}
\newcommand\gitdate{\DTMusedate{gitdate}}
\makeatother

% \setotherlanguage{latin}
% \usepackage{lipsum}

\usepackage{multirow}
\usepackage{graphicx}

\usepackage{lstsmalltalk}

\usepackage{url} % define and apply style to URLs
\def\url@sfstyle{\def\UrlFont{\sf}}
\urlstyle{sf}
\usepackage[unicode,breaklinks,hidelinks]{hyperref} % undecorated hyperlinks

\title{the Square Bracket Associates\titlebreak{}
  \texorpdfstring{\protect\LaTeX}{LaTeX} book class}
\author{Damien Pollet}
\date{\gitdate\titlebreak[\smallskip]{ -- }\protect\gitCommitInfo}

\hypersetup{pdfinfo = {
    Title = {\thetitle},
    Author = {\theauthor},
    Keywords = {LaTeX, document class, book, typography}}}

\newcommand\hrefnote[3][]{% inline linked text, plus URL in footnote
  % \hrefnote[]{URL}{link text}
  \href{#2}{#3}\footnote{\url{#2} #1}}

% a nice horizontal separator
\newcommand{\sectionline}{%
  \begin{center}\nointerlineskip
    \rule{0.233\linewidth}{.7pt}%
  \end{center}}

\begin{document}

%%%%%%%%%%%%%%%%%%%%%%%%%%%%%%%%%%%%%%%%%%%%%%%%%%%%%%%%%%%%%%%%%%%%%%
% Title page and colophon on verso
\maketitle
\pagestyle{titlingpage}
\thispagestyle{titlingpage} % \pagestyle does not work on the first one…

\cleartoverso
{\small

  Copyright 2015 by Damien Pollet.

  The contents of this book are protected under the Creative Commons
  Attribution-ShareAlike 3.0 Unported license.

  You are \textbf{free}:
  \begin{itemize}
  \item to \textbf{Share}: to copy, distribute and transmit the work,
  \item to \textbf{Remix}: to adapt the work,
  \end{itemize}

  Under the following conditions:
  \begin{description}
  \item[Attribution.] You must attribute the work in the manner specified by the
    author or licensor (but not in any way that suggests that they endorse you
    or your use of the work).
  \item[Share Alike.] If you alter, transform, or build upon this work, you may
    distribute the resulting work only under the same, similar or a compatible
    license.
  \end{description}

  For any reuse or distribution, you must make clear to others the
  license terms of this work. The best way to do this is with a link to
  this web page: \\
  \url{http://creativecommons.org/licenses/by-sa/3.0/}

  Any of the above conditions can be waived if you get permission from
  the copyright holder. Nothing in this license impairs or restricts the
  author's moral rights.

  \begin{center}
    \includegraphics[width=0.2\textwidth]{CreativeCommons-BY-SA.pdf}
  \end{center}

  Your fair dealing and other rights are in no way affected by the
  above. This is a human-readable summary of the Legal Code (the full
  license): \\
  \url{http://creativecommons.org/licenses/by-sa/3.0/legalcode}

  \vfill

  \begin{tabular}{@{}c@{\quad}l}
    \multirow{2}{*}{\includegraphics[width=2em]{sba-logo.pdf}}
    & Developed for the books published by Square Bracket Associates. \\
    & \url{http://squarebracketassociates.org} \\[\smallskipamount]
  \end{tabular}
}


\frontmatter
\pagestyle{plain}

%%%%%%%%%%%%%%%%%%%%%%%%%%%%%%%%%%%%%%%%%%%%%%%%%%%%%%%%%%%%%%%%%%%%%%
\chapter*{About this document}

This document is both the showcase and documentation of the SBAbook document
class.
You are of course encouraged to check its code, as a source of concrete examples.

The primary goals of the SBAbook class are:
\begin{itemize}
\item to provide a nice design for the Square Bracket Associates books,
\item to work as an export target for
  \hrefnote{https://github.com/pillar-markup}{Pillar}, the documentation markup
  used in the \hrefnote{http://pharo.org}{Pharo project}.
\end{itemize}
It is therefore meant more as a convenience class than a generic book class.
In fact, SBAbook itself is pretty small; it relies heavily on the
\hrefnote{https://www.ctan.org/pkg/memoir}{Memoir class} and on a selection of
additional packages, which it parameterizes to implement a specific design, from
page layout and typography to semantic markup.

The result is a class that should work well for technical books and should still
be pretty flexible, but which also has a few requirements and makes technical
choices on its own.
This document will thus only describe the specifics of SBAbook and its requirements.


\tableofcontents*

\clearpage
\listoffigures

\mainmatter


%%%%%%%%%%%%%%%%%%%%%%%%%%%%%%%%%%%%%%%%%%%%%%%%%%%%%%%%%%%%%%%%%%%%%%
\chapter{Technical requirements}


\section{Motivations}

Reasons (fonts, packages)

\begin{figure}[tb]
  \caption{A rather large representation of the icon for the Creative Commons
    license we at Square Bracket Associates use for our books.}
  \includegraphics{CreativeCommons-BY-SA}
  \label{fig:cc-by-sa-icon}
\end{figure}


\section{A pretty up-to-date TeXlive}

how to update
important packages


\section{Building with Lua\LaTeX}

how to run, latexmk


%%%%%%%%%%%%%%%%%%%%%%%%%%%%%%%%%%%%%%%%%%%%%%%%%%%%%%%%%%%%%%%%%%%%%%
\chapter{Page layout and design}


\section{Font choices}

In technical writing, and especially using \LaTeX{}, it's tempting to try to use
fonts as a semantic markup of sorts; unfortunately, this only results in a
jumble of too many fonts and uselessly noisy paragraphs.
Granted, a technical book can never be as minimal as a novel typeset in a single
size, single weight roman font, but some restraint should help maintain a clear
visual hierarchy.
We've thus tried to keep to a minimal set of fonts with clearly defined roles.
Thankfully, there has been several impressive open-source fonts released in the
last years, which made several combinations possible; in the end, we settled on
three families in Table~\ref{tab:fontRoles}, which are all packaged in the
\TeX{}live distribution.

The workhorse font is \hrefnote[--- SIL Open Font
License]{http://software.sil.org/gentium/}{Gentium}; it's compact, very legible,
and not as cold as the fonts we usually see in \LaTeX{} documents.
In titles, \hrefnote[--- Apache
License]{https://www.google.com/fonts/specimen/Open+Sans}{Open Sans} gives a
neutral but friendly impression; it is not overpowering and is very legible in
small sizes, making it perfect for captions and page decorations.
Finally, the font for source code excerpts and verbatim text is \hrefnote[---
SIL Open Font License]{https://mozilla.github.io/Fira/}{Fira Mono}.

\begin{table}
  \caption{Fonts used in the document, and their roles}
  \begin{tabular}{ll}
    \toprule
    \textnormal{Gentium -- \textit{Italic}} & Primary, paragraph text \\
    \textsf{Open Sans -- \textbf{Bold}}     & Structural and secondary text: titles, captions \\
    \texttt{Fira Mono}                      & Verbatim text and code \\
    \bottomrule
  \end{tabular}
  \label{tab:fontRoles}
\end{table}


\section{Text layout}

ragged right
spaced paragraphs with no indent
hanging section numbers


\section{Custom semantic markup}

chapterprecis
constraints on content


\section{Source code and listings}

SBAbook provides custom high-level semantic markup for source code, which
relies on the powerful \code{listings} and \code{tcolorbox} packages.


\paragraph{Inline source code}
For short mentions of source code in the middle of the text, use the
\code§\code|...|§ macro.
This macro is an alias to \code§\lstinline§, which works like verbatim: it uses
an arbitrary delimiter.

Using this macro with curly braces like \code§\code{this}§ is possible, but this
is an experimental feature of \code{listings}, and it breaks in some cases; it's
fine when used in paragraph text, but things get weird as soon as it's part of
the argument of some other macro.
Table~\ref{tab:codeDelims} lists a few characters in the ASCII range that are
convenient as delimiters.

\begin{table}
  \caption{Some convenient delimiters for inline code}
  \begin{tabular}{ll@{\qquad}ll}
    \toprule
    \code_\code|...|_ & pipe                      & \code¶\code§...§¶ & section sign \\
    \code|\code_..._| & underscore                & \code§\code¶...¶§ & pilcrow      \\
    \code¡\code!...!¡ & exclamation point         & \code~\code$...$~ & dollar sign  \\
    \code!\code¡...¡! & inverse exclamation point & \code$\code~...~$ & tilde        \\
    \code¿\code?...?¿ & question mark             & \code`\code^...^` & circumflex   \\
    \code?\code¿...¿? & inverse question mark     & \code^\code`...`^ & backquote    \\
    \bottomrule
  \end{tabular}
  \label{tab:codeDelims}
\end{table}


\paragraph{Displayed source code}
For multi-line excerpts of source code that should appear in the flow of the
text, use the \code{displaycode} environment.
You have to specify the language each excerpt of code is written in, as an
argument; any language known to the \code{listings} package works.

\begin{displaycode}{[Visual]Basic}
10 PRINT "Hello SBAbook!"
20 PRINT "This is a paragraph-level listing."
25 PRINT "UTF-8 in Basic: élèves français, Ελλάδα, Здравствуй, мир!"
30 GOTO 10
\end{displaycode}


\paragraph{Referenced listings}
If you need a numbered source code listing that is referenced in the table of
contents, use the \code{listing} environment, providing a descriptive caption as
a second mandatory argument.
To reference that listing from the text, the label must be provided as
\code{label=something} in the optional argument, as in
listing~\ref{lst:fooNewWith} below.
Similarly, use the \code{list text} option from \code{tcolorbox} if the caption
should be worded differently in the table of contents.

\begin{listing}[label=lst:fooNewWith]{Smalltalk}
{A factory method in class \code{Foo}}
Foo class>>with: parameter
  ^ self new
    initializeWith: parameter
\end{listing}


\paragraph{Floating listings}
To make a floating listing, simply add the \code{float} option in the
environment's optional parameter.
In case you really need to specify the placement, the usual specifiers can also
be passed: \code{float=htbp}.


\section{Packages and conventions}



%%%%%%%%%%%%%%%%%%%%%%%%%%%%%%%%%%%%%%%%%%%%%%%%%%%%%%%%%%%%%%%%%%%%%%
\clearpage
% \textlatin{\lipsum[2-7]}

\end{document}
